\section{Geometric meaning of transpose}

Let's consider two vectors, $a$ and $b$ and an action $M$.
We apply the action $M$ to the first vector $a$ and
consider the scalar product $\langle Ma, b\rangle$.
What should be the equivalent action $N$ on $b$ that whould result
in exactly the same scalar product?
This action $N$ can be found from the equation
\[
\langle Ma, b \rangle = \langle a, Nb \rangle.
\]

We call the action $N$ as transpose of the action $M$.
Our definition explains immediately why it is called transpose:
the letter $M$ has changed its position.
This definition is consistent with classical one, as
\[
\langle Ma, b\rangle = (Ma)^T b = a^T (M^T b) = \langle a, M^T b \rangle
\]

Let's find some transposes using geometric approach.

Example 1.

Consider the action $R$ on $\mathbb{R}^2$ that rotates arbitrary vector $a$ clockwise by the angle $\alpha$.
What is the equivalent action on vector $b$? We can compensate the clockwise rotation $R$ of $a$
by a counterclockwise rotation of $b$. Under both rotations the lengths are preserved
and the angle between $a$ and $b$ is changed in the same way.

So the transpose of clockwise rotation is the counterclockwise rotation by the same angle.

Example 2.

Consider the action $U$ on $\mathbb{R}^3$ that reflects arbitrary vector $a$ across plane $\alpha$.
The equivalent action on the other vector $b$ would be this reflection $U$ itself!
Consider two scalar products, $\langle Ua, b \rangle$ and $\langle a, Ub \rangle$.
We see that vectors $a$ and $Ub$ are reflected vectors $Ua$ and $b$.
So the scalar products are equal.

The transpose of reflection is the same reflection, so $U^T=U$.

Example 3.


Consider the action $H$ on $\mathbb{R}^3$ that projects arbitrary vector $a$ on the plane $\alpha$.
Let $\hat a$ and $\hat b$ be the projections of vectors $a$ and $b$.
Scalar product is a linear function, and $\hat a$ is orthogonal to $b - \hat b$, so
\[
\langle Ha, b \rangle = \langle \hat a, \hat b + (b-\hat b) \rangle = \langle \hat a, \hat b \rangle
\]
By the same reasoning $\langle a, Hb \rangle = \langle \hat a, \hat b \rangle$.

And the transpose of projection is the same projection, $H^T=H$.

The equality $\langle Ha, b\rangle = \langle \hat a, \hat b \rangle$ holds only for orthogonal projection,
and not for the oblique one.
So $H^T \neq H$ for oblique projections.


The idea of a transpose may be also illustrated if one knows about singular value decomposition.
From singular value decomposition $A = U\Sigma V^T$ every action $A$ can be decomposed in rotation followed by scaling and
second rotation.
The transpose of rotation is inverse rotation. The transpose of scaling is the same scaling.
So the transpose of every action $A$ is the inverse of second rotation, the same scaling and inverse of first rotation.
It follows by the way that $\det A = \det A^T$ as rotations do not change volume and scaling is unaffected by transpose.

TODO: add pictures

TODO: Why $H^2=H$ and $H^T=H$ is sufficient for projection?

\url{https://math.stackexchange.com/questions/598258}
